\documentclass[11pt, oneside]{article}   	% use "amsart" instead of "article" for AMSLaTeX format
\usepackage{geometry}                		% See geometry.pdf to learn the layout options. There are lots.
\geometry{letterpaper}                   		% ... or a4paper or a5paper or ... 
%\geometry{landscape}                		% Activate for for rotated page geometry
\usepackage[parfill]{parskip}    		% Activate to begin paragraphs with an empty line rather than an indent
\usepackage{graphicx}				% Use pdf, png, jpg, or eps§ with pdflatex; use eps in DVI mode
								% TeX will automatically convert eps --> pdf in pdflatex		
\usepackage{amssymb}
%\usepackage{amsmath}

\setlength{\topmargin}{-0.5in}
\setlength{\textheight}{9in}
\setlength{\textwidth}{7in}
\setlength{\oddsidemargin}{-.25in}

\title{Permutation Generation with Heap's Algorithm and the Steinhaus-Johnson-Trotter Algorithm}
\author{Ryan Bernstein \\ Ruben Niculcea \\ Levi Schoen}
\date{}							% Activate to display a given date or no date

\begin{document}
\maketitle

\tableofcontents
\newpage

\section{Heap's Algorithm}

\subsection{Time Complexity}

The significant portion of our implementation of Heap's Algorithm is as follows:
\begin{verbatim}
  3 def heaps(set, time=False):
  4   def innerHeaps(n, set):
  5     if n == 1:
  6       print set
  7     else:
  8       for i in range(n):
  9         innerHeaps(n-1, set)
 10         if n % 2 == 1: #odd number
 11           j = 0
 12         else:
 13           j = i
 14         set[j], set[n-1] = set[n-1], set[j] #swap
\end{verbatim}

To find the total running time of our algorithm, we'll look at the number of times the following operations are executed:
\begin{itemize}
	\item Swaps, an $O(1)$ operation
	\item Comparisons, an $O(1)$ operation
	\item Assignments, an $O(1)$ operation, and
	\item Processes (\texttt{print}s in our implementation), an $O(n)$ operation
\end{itemize}

From the code above, we can count how many times each of these operations is performed, \emph{not} including work done within the recursive calls. To better facilitate our analysis, we will also include the number of recursive calls.

\subsubsection{Swaps}

The swap operation appears only once in our implementation, on line 14. This line is never executed when $n=1$; if $n > 1$, the loop containing it executes $n$ times.

\subsubsection{Comparisons}

A comparison appears on line 5 regardless of the current value of $n$. Another comparison appears within the loop on line 10, which executes $n$ times for all $n > 1$. The second comparison also includes a division for modular arithmetic; we will also consider this an $O(n)$ operation.

\subsubsection{Assignments}

Not counting the assignments that appear within the swaps, a single assignment occurs on either line 11 or line 13. This is dependent on the result of the comparison on line 10.

\subsubsection{Processes}

We only process a permutation in the base case of our recursion, when $n = 1$. Thus we execute a process once when $n = 1$ and 0 times in any other case.

Our total number of operations (not counting recursion) can be represented by the following table:

\begin{center}
\begin{tabular}{|c|c|c|c|c|c|}
	\hline
	$n$ & 1 & 2 & 3 & 4 & 5 \\
	\hline
	Swaps & 0 & 2 & 3 & 4 & 5 \\
	\hline
	Comparisons & 1 & 3 & 4 & 5 & 6 \\
	\hline
	Assignments & 0 & 2 & 3 & 4 & 5 \\
	\hline
	Processes & 1 & 0 & 0 & 0 & 0 \\
	\hline
	Recursive calls & 0 & 2 & 3 & 4 & 5 \\
	\hline
\end{tabular}
\end{center}

\subsection{Space Complexity}

\end{document}  